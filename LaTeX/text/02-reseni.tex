\chapter[Sledování a predikce pohybu těles na oběžné dráze Země]{Sledování a predikce pohybu těles na oběžné dráze Země}

Objekty, které se pohybují vesmírem kolem Země jsou sledovány organizacemi \cite{wiki:derbis}:
\begin{itemize}
    \item \zkratka{USSTRATCOM}(součást \zkratka{DoD})
    \item \zkratka{ESA}
    \item \zkratka{Fraunhofer-FHR}
    \item \zkratka{JPL}(součast \zkratka{NASA})
    \item \zkratka{MIT}
    \item \zkratka{EISCAT}
    \item \zkratka{USAF}
\end{itemize}

Ke sledování se používají pozemní radary, lidary, pozemní  a vesmírné teleskopy. Mezi sledované objekty patří umělé družice, pozůstatky raket a jiný vesmírný odpad. Nejrozsáhlejší katalog stavu družic udržuje Ministerstvo obrany Spojených států (\zkratka{DoD}) s názvem Space Object Catalog. Civilní varianta této databáze je provozována organizaci \zkratka{NASA}. Tyto databáze se udržují pomocí různých modelů orbitální mechaniky. Pohyby družic jsou analyticky vypočteny pomocí teorie všeobecných perturbací. Prvky dráhy této teorie jsou publikovány ve formátu \zkratka{NASA}/NORAD \zkratka{TLE}. \cite{wiki:US_space_surv}.

Proč se zabývat přesnou polohou družice? Pro korekci dopplerovského posuvu frekvence musíme znát polohu naši pozemní stanice, polohu družice a předpovědět její polohu. K tomu pozdějšímu nám právě napomáhají prvky dráhy ze souborů \zkratka{TLE} využívajíc model \zkratka{SGP4} \cite{wiki:TLE}.

\chapter{}
