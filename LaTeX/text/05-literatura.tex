% Pro sazbu seznamu literatury použijte jednu z následujících možností

%%%%%%%%%%%%%%%%%%%%%%%%%%%%%%%%%%%%%%%%%%%%%%%%%%%%%%%%%%%%%%%%%%%%%%%%%
%1) Seznam citací definovaný přímo pomocí prostředí literatura / thebibliography

\begin{literatura}{99}

\bibitem{wiki:amateur_sat}
    \emph{Amateur radio satellite}. In: Wikipedia: the free encyclopedia\/ [online].
    San Francisco (CA): Wikimedia Foundation, 2001- [cit. 2017-12-10].
    Dostupné z~URL:\\
    <\url{https://en.wikipedia.org/wiki/Amateur_radio_satellite}>.

\bibitem{wiki:AMSAT}
    \emph{Amateur radio satellite}. In: Wikipedia: the free encyclopedia\/ [online].
    San Francisco (CA): Wikimedia Foundation, 2001- [cit. 2017-12-10].
    Dostupné z~URL:\\
    <\url{https://en.wikipedia.org/wiki/AMSAT}>.

\bibitem{book:ARRL_handbook}
    PUBLISHED BY AMERICAN RADIO RELAY LEAGUE.
    \emph{The ARRL handbook for radio communications 2011.} 88th ed. Newington, CT: American Radio Relay League, 2010. ISBN 9780872590953.

\bibitem{wiki:LEO}
    \emph{Low Earth orbit}. In: Wikipedia: the free encyclopedia\/ [online].
    San Francisco (CA): Wikimedia Foundation, 2001- [cit. 2017-12-10].
    Dostupné z~URL:\\
    <\url{https://en.wikipedia.org/wiki/Low_Earth_orbit}>.

\bibitem{wiki:derbis}
    \emph{Space derbis}. In: Wikipedia: the free encyclopedia\/ [online].
    Dostupné z~URL:\\
    <\url{https://en.wikipedia.org/wiki/Space_debris}>.

\bibitem{wiki:US_space_surv}
    \emph{United States Space Surveillance Network}. In: Wikipedia: the free encyclopedia\/ [online].
    Dostupné z~URL:\\
    <\url{https://en.wikipedia.org/wiki/United_States_Space_Surveillance_Network}>.

\bibitem{wiki:TLE}
    \emph{Two-line element set}. In: Wikipedia: the free encyclopedia\/ [online].
    Dostupné z~URL:\\
    <\url{https://en.wikipedia.org/wiki/Two-line_element_set}>.



%\bibitem{CSN_ISO_690-2011}
%    \emph{ČSN ISO 690 (01 0197) Informace a dokumentace -- Pravidla pro bibliografické odkazy a citace informačních zdrojů.}
%    40 stran. Praha: Český normalizační institut, 2011.
%
%\bibitem{CSN_ISO_7144-1997}
%    \emph{ČSN ISO 7144 (010161) Dokumentace -- Formální úprava disertací a podobných dokumentů.}
%    24 stran. Praha: Český normalizační institut, 1997.
%
%\bibitem{CSN_ISO_31-11}
%    \emph{ČSN ISO 31-11 Veličiny a jednotky -- část 11: Matematické znaky a značky používané ve fyzikálních vědách a v~technice.}
%    Praha: Český normalizační institut, 1999.
%
%\bibitem{BiernatovaSkupa2011:CSNISO690komentar}
%    BIERNÁTOVÁ, O., SKŮPA, J.:
%    \emph{Bibliografické odkazy a citace dokumentů dle ČSN ISO 690 (01 0197) platné od 1.\,dubna 2011}\/ [online].
%    2011, poslední aktualizace 2.\,9.\,2011 [cit. 19.\,10.\,2011].
%    Dostupné z~URL:
%    \(<\)\url{http://www.citace.com/CSN-ISO-690.pdf}\(>\)
%%    \(<\)\href{http://www.boldis.cz/citace/citace.html}{http://www.boldis.cz/citace/citace.html}\(>\).
%
%\bibitem{pravidla}
%    \emph{Pravidla českého pravopisu}.
%    Zpracoval kolektiv autorů. 1.\ vydání.
%    Olomouc: FIN PUB\-LISH\-ING, 1998. 575 s. ISBN 80-86002-40-3.
%
%\bibitem{Walter1999}
%  WALTER, G.\,G.; SHEN, X.
%  \emph{Wavelets and Other Orthogonal Systems}.
%  2. vyd. Boca Raton: Chapman\,\&\,Hall/CRC, 2000. 392~s. ISBN 1-58488-227-1
%
%\bibitem{Svacina1999IEEE}
%  SVAČINA, J.
%  Dispersion Characteristics of Multilayered Slotlines -- a Simple Approach.
%  \emph{IEEE Transactions on Microwave Theory and Techniques},
%  1999, vol.\,47, no.\,9, s.\,1826--1829. ISSN 0018-9480.
%
%\bibitem{RajmicSysel2002}
%    RAJMIC, P.; SYSEL, P.
%    Wavelet Spectrum Thresholding Rules.
%    In \emph{Proceedings of the International Conference Research in Telecommunication Technology},
%    Žilina: Žilina University, 2002. s.\,60--63. ISBN 80-7100-991-1.

\end{literatura}


%%%%%%%%%%%%%%%%%%%%%%%%%%%%%%%%%%%%%%%%%%%%%%%%%%%%%%%%%%%%%%%%%%%%%%%%%
%%2) Seznam citací pomocí BibTeXu
%% Při použití je nutné v TeXnicCenter ve výstupním profilu aktivovat spouštění BibTeXu po překladu.
%% Definice stylu seznamu
%\bibliographystyle{unsrturl}
%% Pro českou sazbu lze použít styl czechiso.bst ze stránek
%% http://www.fit.vutbr.cz/~martinek/latex/czechiso.tar.gz
%%\bibliographystyle{czechiso}
%% Vložení souboru se seznamem citací
%\bibliography{text/literatura}
%
%% Následující příkaz je pouze pro ukázku sazby literatury při použití BibTeXu.
%% Způsobí citaci všech zdrojů v souboru odkazy.bib, i když nejsou citovány v textu.
%\nocite{*}
