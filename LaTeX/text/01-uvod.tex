\chapter*{Úvod}
\phantomsection
\addcontentsline{toc}{chapter}{Úvod}

Radiokomunikace prošla ohromným vývojem v průběhu druhé polovice 20. století. Vypuštění první umělé družice Sputnik-1 v roce 1957 datujeme začátek vesmírného věku. Od té doby se otevřeli nové možnosti radioamatérův věnovat se svému koníčku, nebo výzkumu.

Necelé čtyři roky po vypuštění první umělé družice Země se svět dočkal první amatérské umělé družice sestrojeného v rámci projektu \zkratka{OSCAR} \cite{wiki:amateur_sat}, kterého nástupnickou organizaci se stal \zkratka{AMSAT} \cite{wiki:AMSAT}. Družice OSCAR I byla vypuštěná na nízkou oběžnou dráhu jako druhotný náklad, který vyžíval rezervy nosnosti rakety  Thor DM-21 Agena-B. Tento způsob dopravy byl zvolen z ekonomických důvodů a je dodnes používán.

Obecně se amatérské družice nasazují na nízkou oběžnou dráhu (Low Earth Orbit -- LEO) \cite{book:ARRL_handbook}. Výhodou dráhy tohoto typu je jejich finanční nenáročnost, co je zčásti způsobená vlastností plynoucího z nazvu oběžné dráhy, výškou orbitu, který se pohybuje od $300\,\mathrm{km}$ do $2000\,\mathrm{km}$. Nedostatkem nízké oběžné dráhy je vysoká relativní rychlost družice vůči pozemní stanici, která přibližně $7{,}8\,\mathrm{\nicefrac{km}{s}}$ \cite{wiki:LEO} jehož důsledkem je rádiový signál značně postižen dopplerovským posuvem kmitočtu, kterého velikost dosahuje i $26\,\mathrm{ppm}$.

\section*{Cíle práce}
\phantomsection
\addcontentsline{toc}{section}{Cíle práce}
Na palubě družic se nacházejí přístroje určené k měření elektrických i neelektrických veličin charakterizující stav kosmické lodi. Tyto údaje jsou v jistých časových okamžicích odvysílané směrem na Zem.

Cílem této práce je zpracování přijatého signálu postiženého dopplerovským posuvem, demodulace a dekódování signálu a následná archivace dat pro její budoucí analyzování .
