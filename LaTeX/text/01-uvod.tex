\chapter*{Úvod}
\phantomsection
\addcontentsline{toc}{chapter}{Úvod}

Umělé družice obíhající kolem Země jsou systémy, s kterými poslední fyzické kontakty lidí vznikají těsně před její vypuštěním. Prakticky to znamená praktickou nemožnost provádění jakékoliv činnosti na zařízení v místě působení činnosti umělé družice. Proto hraje důležitou roli při provozování umělých družic  dálkový sběr naměřených údajů senzorů, tzv. telemetrických dat, umělých družic za účelem vyhodnocení její stavu.

V této práci je vytvářená snaha o automatické zpracování telemetrických dat amatérských umělých družic Parkinson NO-83 a NO-84. Protože tyto družice mají svou dráhu na nízké oběžné dráze, je jejich signál silně postižen dopplerovským posuvem. Pro správní demodulaci signálu se tento posuv kmitočtu signálu musí korigovat, a kvůli číslicové povaze modulovaného signálu musí být táto korekce prováděná bez vzniku fázové diskontinuity.

Dělení dokumentu do kapitol a podkapitol je tvořen ze záměrem porozumění a sledování vzniku řešení zadaného a problému. Ke kompenzaci dopplerovského posuvu kmitočtu signálu musíme znát relativní rychlost družice vzhledem ke pozemní stanici. Pro výpočet této rychlosti musíme znát naši polohu, predikovat pohyb umělé družice a tvar její dráhy. Struční úvod do této problematiky na popsán v úseku \ref{sec:position}. V následující úseku \ref{sec:doppler} se věnujeme výpočtu relativní rychlosti umělé družice vzhledem k pozemní stanici nutnou pro výpočet velikosti dopplerovského posuvu.

V kapitole \ref{chap:python3} se prezentuje konkrétní řešení, včetně vývojových diagramů a popisu fungování jednotlivých skript v jazyce Python 3.

V příloze jsou uvedeny zdrojové kódy prezentovaných modulů.
