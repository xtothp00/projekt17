\autor[Bc.]{Péter}{Tóth}
\vedouci[Ing.]{Tomáš}{Urbanec }[Ph.D.]
\nazev{Název studentské práce}{Title of Student's Thesis}
%\oponent[doc.\ Mgr.]{Křestní}{Příjmení}[Ph.D.]
\oborstudia{Teleinformatika}{Teleinformatics}
\fakulta{Fakulta elektrotechniky a komunikačních technologií}{Faculty of Electrical Engineering and Communication}
\ustav{Ústav radioelektroniky}{Department of Radio Electronics}
\logofakulta[loga/FEKT_zkratka_barevne_PANTONE_CZ]{loga/UTKO_color_PANTONE_CZ}
\rok{2017}
\datum{19.\,12.\,2017} % Datum se uplatní pouze v prezentaci k obhajobě
\misto{Brno}
\abstrakt{Táto práce se zaobírá zpracováním přijatých signálů z amatérských družic NO-83 a NO-84 ParkinsonSat na nízké oběžné dráze (Low Earth Orbit – LEO) vysílajících telemetrické údaje v pásmu $70\,\mathrm{cm}$ vln. které jsou postiženy dopplerovským posuvem kmitočtu. Kvůli povaze oběžné dráhy a kmitočtu vysílání, přijatý signál je znatelně poškozen dopplerovským posuvem kmitočtu, který se musí kompenzovat pro pozdější potřeby demodulace.
}{This project work is dealing with processing of received radio signals of LEO satellites NO-83 and NO-84 ParkinsonSat transmitting in the 70--centimeter band. The nature of this kind of setup makes the received signal bearing a large amount of Doppler shift, which needs to be compensated in order to demodulate it.
}
\klicovaslova{korekce dopplerského posuvu, NO-83, BRICsat, NO-84, PSat, družice LEO, archiv dat TLE}
  {doppler shift correction, NO-83, BRICsat, NO-84, PSat, LEO satellites, TLE data archive}
\podekovanitext{Rád bych poděkoval vedoucímu diplomové práce panu Ing.~Tomášu Urbanci, Ph.D.\ za odborné vedení, konzultace, trpělivost a podnětné návrhy k~práci.}
